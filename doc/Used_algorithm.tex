\makeatletter
\def\input@path{{macros/}}
\makeatother 

\documentclass[12pt]{article}

\usepackage[utf8]{inputenc}
\usepackage[UKenglish]{babel}

%\usepackage{colortbl}
\usepackage{graphics,graphicx}
\usepackage{subfigure}
\graphicspath{{figs/}}
\usepackage[usenames,dvipsnames]{color}
\usepackage{listings}
\usepackage{fancyvrb}
\usepackage{bm}
\usepackage[math]{easyeqn}

\title{Astro DOC}
\author{Enrico Bertolazzi}
\date{2023}			



\begin{document}
\maketitle

% ___           _              _   _      _ 
%| __|__ _ _  _(_)_ _  ___  __| |_(_)__ _| |
%| _|/ _` | || | | ' \/ _ \/ _|  _| / _` | |
%|___\__, |\_,_|_|_||_\___/\__|\__|_\__,_|_|
%       |_|                                 
%                    _ _           _          
% __ ___  ___ _ _ __| (_)_ _  __ _| |_ ___ ___
%/ _/ _ \/ _ \ '_/ _` | | ' \/ _` |  _/ -_|_-<
%\__\___/\___/_| \__,_|_|_||_\__,_|\__\___/__/

\section{Modified equinoctial coordinates}

The problem has been formulated with modified equinoctial coordinates.
In this way, the dynamics of the system can be described in terms of state variables:
%%%
\begin{EQ}
  \pmatrix{\bm{y}^{T},m}=\pmatrix{p,f,g,h,k,L,m}
\end{EQ}
%%%
and the control variables in spacecraft-attached Frenet reference frame:
%%%
\begin{EQ}
  \bm{T}^{T}=\pmatrix{T_{r},T_{t},T_{n}}
\end{EQ}
%%%
The equations of motion for the spacecraft with variable thrust can be stated as:
%%%
\begin{EQ}[rcl]\label{eq:equi}
  \dot{\bm{y}} &=& \bm{A}(\bm{y})\frac{\bm{T}}{m} + \bm{b}, \\
  \dot{m}      &=& -\frac{\norm{\bm{T}}}{g_{E}I_{sp}} \\
  \abs{\bm{T}} &=& \sqrt{T_{r}^{2}+T_{t}^{2}+T_{n}^{2}} \leq T_{max}
\end{EQ}
%%%
The equinoctial dynamics are in fact defined by:
%%%
\begin{EQ}[rcl]
  \bm{A}&=&
  \dfrac{1}{q}
  \sqrt{\dfrac{p}{\mu}}
  \pmatrix{
    0 &  2p &  0\\
    q\sin L  & \{(q+1)\cos L+f\} & -g\{h\sin L-k\cos L\} \\
    -q\cos L & \{(q+1)\sin L+g\} & f\{h\sin L-k\cos L\} \\
    0 & 0 & \dfrac{s^{2}\cos L}{2}\\
    0 & 0 & \dfrac{s^{2}\sin L}{2}\\
    0 & 0 & h\sin L-k\cos L
  }
  \\
  \bm{b}^{T}&=&\pmatrix{
    0 & 0 & 0 & 0 & 0 & \sqrt{\mu p}\left(\dfrac{q}{p}\right)^{2}
  }
\end{EQ}
%%%
where:
%%%
\begin{EQ}
  q = 1 + f\cos L + g\sin L, \qquad
  r = \frac{p}{q},\qquad
  \alpha^{2} = h^{2} - k^{2}, \\
  \chi = \sqrt{h^{2} + k^{2}}, \qquad
  s^{2}=1+\chi^{2}
\end{EQ}
%%%
The equinoctial coordinates y are related to the Cartesian state $(r, v)$ according to the expressions:
%%%
\begin{EQ}[rcl]
  \bm{r}(p, f, g, h, k, L) &=&\dfrac{r}{s^{2}}\pmatrix{
     (1+\alpha^{2})\cos L + 2hk\sin L\\
     (1-\alpha^{2})\sin L + 2hk\cos L\\
     2\left( h\sin L -k\cos L \right)\\
  }
  \\
  \bm{v}(p, f, g, h, k, L) &=&
  \dfrac{1}{s^{2}}\sqrt{\frac{\mu}{p}}
  \pmatrix{ 
     -\left( (1+\alpha^{2})(g+\sin L) - 2hk(f+\cos L)\right)\\
     -\left( (\alpha^{2}-1)(f+\cos L) + 2hk(g+\sin L)\right)\\
     2\left( h(f+\cos L) + k(g+\sin L)\right)\\
  }
\end{EQ}
%%%
The vector $\bm{T}$ is expressed in the Frenet rotating radial
frame whose principal axes are defined by:
%%%
\begin{EQ}
  \bm{Q}_{r} = \pmatrix{
    \bm{i}_{r} & \bm{i}_{t}	& 	\bm{i}_{n}
   }
   = \pmatrix{
       \dfrac{\bm{r}}{\norm{\bm{r}}} &
       \dfrac{(\bm{r} \times \bm{v}) \times \bm{r}}{\norm{\bm{r} \times \bm{v}}\norm{\bm{r}}}&
       \dfrac{\bm{r} \times \bm{v}}{\norm{\bm{r} \times \bm{v}}}
  }
\end{EQ}
%%%
The choice of these coordinates is made because when the thrust is zero $\textbf{T}^{T} = [ 0, \; 0, \; 0 ]$ the first five equations are simply $\dot{p} = \dot{f} = \dot{g} = \dot{h} = \dot{k} = 0$, which implies that the elements are constant. Moreover the behaviour of the controls becomes more regular. These two features allow a good formulation of the optimal control problem and they simplify the numerical solution.


\newcommand{\SP}{\mathrm{d}}
\section{Equinoctial coordinates with $\SP=\sqrt{p}$}

%%%
\begin{EQ}
  \pmatrix{\bm{y}^{T},m}=\pmatrix{\SP,f,g,h,k,L,m},\qquad
  \bm{T}^{T}=\pmatrix{T_{r},T_{t},T_{n}}
\end{EQ}
%%%
The equations of motion for the spacecraft with variable thrust can be stated as:
%%%
\begin{EQ}\label{eq:equi}
  \dot{\bm{y}} = \bm{A}(\bm{y})\frac{\bm{T}}{m} + \bm{b},\quad
  \dot{m}      = -\frac{\norm{\bm{T}}}{g_{E}I_{sp}}, \quad
  \abs{\bm{T}} = \sqrt{T_{r}^{2}+T_{t}^{2}+T_{n}^{2}} \leq T_{max}
\end{EQ}
%%%
Usando la relazione
%%%
\begin{EQ}
   \dfrac{dp}{dt} =
   \dfrac{db^2}{dt} =
   2b\dfrac{db}{dt}
\end{EQ}
%%%
\begin{EQ}[rcl]
  \bm{A}&=&
  \dfrac{1}{\sqrt{\mu}}
  \dfrac{b}{q}
  \pmatrix{
    0 & b & 0\\
     q\sin L & (q+1)\cos L+f & -g\Big(h\sin L-k\cos L\Big) \\
    -q\cos L & (q+1)\sin L+g & f\Big(h\sin L-k\cos L\Big) \\
    0 & 0 & \dfrac{s^{2}}{2}\cos L\\
    0 & 0 & \dfrac{s^{2}}{2}\sin L\\
    0 & 0 & h\sin L-k\cos L
  }
  \\
  \bm{b}^{T}&=& \sqrt{\mu}\pmatrix{
    0 & 0 & 0 & 0 & 0 &\dfrac{q^2}{b^3}
  }
\end{EQ}
%%%
where:
%%%
\begin{EQ}
  q = 1 + f\cos L + g\sin L, \qquad
  r = \frac{p}{q}= \frac{b^2}{q},\qquad
  \alpha^{2} = h^{2} - k^{2}, \\
  \chi = \sqrt{h^{2} + k^{2}}, \qquad
  s^{2}=1+\chi^{2}
\end{EQ}
%%%
The equinoctial coordinates y are related to the Cartesian state $(r, v)$ according to the expressions:
%%%
\begin{EQ}[rcl]
  \bm{r}(b, f, g, h, k, L) &=&\dfrac{r}{s^{2}}\pmatrix{
     (1+\alpha^{2})\cos L + 2hk\sin L\\
     (1-\alpha^{2})\sin L + 2hk\cos L\\
     2\left( h\sin L -k\cos L \right)\\
  }
  \\
  \bm{v}(b, f, g, h, k, L) &=&
     \frac{\sqrt{\mu}}{bs^2}\pmatrix{ 
     -\left( (1+\alpha^{2})(g+\sin L) - 2hk(f+\cos L)\right)\\
     -\left( (\alpha^{2}-1)(f+\cos L) + 2hk(g+\sin L)\right)\\
      2\left( h(f+\cos L) + k(g+\sin L)\right)\\
  }
\end{EQ}
%%%


\section{Lambert Maneuvre}

\begin{itemize}
  \item mean anomaly difference 
  \begin{EQ}[rcl]
    M_2(t_2)-M_1(t_1) &=& n(t_2-t_1)
    =
    \sqrt{\dfrac{\mu}{a^3}}(t_2-t_1)
    \\
    t_2-t_1 
    &=& 
    \sqrt{\dfrac{a^3}{\mu}}
    \left[
      E_2-E_1-e(\sin E_2-\sin E_1)
    \right]
    \\
    &=& 
    2\sqrt{\dfrac{a^3}{\mu}}
    \left[
      \dfrac{E_2-E_1}{2}
      -e\left(\sin\left(\dfrac{E_2-E_1}{2}\right)
              \cos\left(\dfrac{E_2+E_1}{2}\right)\right)
    \right]
    \\
  \end{EQ}
  \item define
  %%%%
  \begin{EQ}
     A=\dfrac{E_2-E_1}{2},\qquad
     \cos B = e\cos\left(\dfrac{E_2+E_1}{2}\right)
  \end{EQ}
  %%%
  \item and
  %%%
  \begin{EQ}
    t_2-t_1 = 2\sqrt{\dfrac{a^3}{\mu}}(A-\sin A \cos B)
  \end{EQ}
  %%%
  \item substitute $\alpha=A+B$, $\beta=B-A$ and using
  %%%
  \begin{EQ}
    \sin\left(\frac{\alpha}{2}\right)^2=\frac{s}{2a},\qquad
    \sin\left(\frac{\beta}{2}\right)^2=\frac{s-c}{2a},
  \end{EQ}
  %%%
  where
  %%%
  \begin{EQ}
     2s=r_1+r_2+c,\qquad
     c^2 = \norm{\bm{r}_1-\bm{r}_2}=r_1^2 + r_2^2 - 2r_1r_2\cos\theta
  \end{EQ}
  %%%
  \item obtain a single equation in the unknown $a$
  %%%
  \begin{EQ}
     t_2-t_1 = 
     \sqrt{\dfrac{a^3}{\mu}}
     \left[
       (\alpha-\sin\alpha)-
       (\beta-\sin\beta)
     \right]
  \end{EQ}
\end{itemize}
%%%
Eccentricity equation:
%%%
\begin{EQ}[rcl]
  F(E) &=& E(t)- e \sin E(t) - M(t)=0\\
  F(a) &=& \sqrt{\dfrac{a^3}{\mu}}
         \left[
           (\alpha(a)-\sin\alpha(a))-
           (\beta(a)-\sin\beta(a))
         \right]
         -t
         =0\\
  F(x) &=& \phi(x)+\phi(y(x,\tau))+N\pi-\tau = 0,\\
  \phi(u) &=& \cot^{-1}\left(\dfrac{u}{\sqrt{1-u^2}}\right)
  -\dfrac{1}{3u}(2+u^2)\sqrt{1-u^2} \\
  \tau &=& 4t\sqrt{\dfrac{\mu}{m^3}},\qquad
  m=\norm{\bm{r}_1}+\norm{\bm{r}_2}+\norm{\bm{r}_1-\bm{r}_2}
\end{EQ}

\section{Optimal two-impulse rendezvous using constrained multiple-revolution Lambert solutions}
\paragraph{Gang Zhang, Di Zhou, Daniele Mortari}

Maneuver from $(\bm{r}_0,\bm{v}_0)$ a $(\bm{r}_1,\bm{v}_1)$ 
with free time minimizing $\Delta V$:
%%%
\begin{EQ}
   \Delta V = \Delta V_0 + \Delta V_1 = \norm{\bm{v}_0-\bm{w}_0}+\norm{\bm{v}_1-\bm{w}_1}
\end{EQ}
%%%
maneuver is determined from $(\bm{r}_0,\bm{w}_0)$ to $(\bm{r}_1,\bm{w}_1)$.
Solution can be written as
%%%
\begin{EQ}
   \cases{
      \bm{w}_0 = \omega_c \hat{\bm{c}}+\omega_\rho\hat{\bm{r}}_0 & \\
      \bm{w}_1 = \omega_c \hat{\bm{c}}-\omega_\rho\hat{\bm{r}}_1 & \\
   }
   \\
   c = \norm{\bm{r}_1-\bm{r}_0},\qquad
   \hat{\bm{c}} = \dfrac{\bm{r}_1-\bm{r}_0}{c},\\
   r_0 = \norm{\bm{r}_0}, \qquad
   \hat{\bm{r}}_0 = \dfrac{\bm{r}_0}{r_0},\qquad
   r_1 = \norm{\bm{r}_1}, \qquad
   \hat{\bm{r}}_1 = \dfrac{\bm{r}_1}{r_1},
\end{EQ}
%%%
moreover 
%%%
\begin{EQ}\label{eq:omegac}
   \omega_c = \dfrac{c\sqrt{\mu p}}{r_0 r_2\sin\theta},
   \quad
   \omega_\rho = \sqrt{\dfrac{\mu}{p}}\dfrac{1-\cos\theta}{\sin\theta},
   \quad
   K = \omega_c \omega_\rho = \dfrac{\mu c}{r_0r_1+\bm{r}_0\cdot\bm{r}_1}
\end{EQ}
%%%
so that
%%%
\begin{EQ}[rcl]
   \bm{w}_0-\bm{v}_0 &=&
   \omega_c\hat{\bm{c}}+(K/\omega_c)\hat{\bm{r}}_0-\bm{v}_0
   \\
   \bm{v}_1-\bm{w}_1 &=& 
   \bm{v}_1-\omega_c\hat{\bm{c}}+(K/\omega_c)\hat{\bm{r}}_1
\end{EQ}
%%%
and
%%%
\begin{EQ}[rcl]
   \norm{\Delta V_0}^2 &=& 
   \omega_c(\omega_c-2(\bm{v}_0\cdot \hat{\bm{c}}))
   + 2K(\hat{\bm{c}}\cdot\hat{\bm{r}}_0)
   +\norm{\bm{v}_0}^2
   +
   \dfrac{K}{\omega_c}
   \left(
    \dfrac{K}{\omega_c} 
   -2(\bm{v}_0\cdot\hat{\bm{r}}_0)
   \right)
   \\
   \norm{\Delta V_1}^2 &=& 
   \omega_c(\omega_c-2(\bm{v}_1\cdot \hat{\bm{c}}))
   -2K(\hat{\bm{c}}\cdot\hat{\bm{r}}_1)
   +\norm{\bm{v}_1}^2
   +
   \dfrac{K}{\omega_c}
   \left(
    \dfrac{K}{\omega_c} 
   -2(\bm{v}_1\cdot\hat{\bm{r}}_1)
   \right)
   %+ \dfrac{K^2}{\omega_c^2}
   %+2\dfrac{K}{\omega_c} (\bm{v}_1\cdot\hat{\bm{r}}_1)
\end{EQ}
%%%
and derivatives
%%%
\begin{EQ}[rcl]
   \dfrac{\mathrm{d}\norm{\Delta V_i}}{\mathrm{d}\omega_c}
   &=&\dfrac{1}{2\norm{\Delta V_i}}\dfrac{\mathrm{d}\norm{\Delta V_i}^2}{\mathrm{d}\omega_c},
   \\
   \dfrac{\mathrm{d}^2\norm{\Delta V_i}}{\mathrm{d}\omega_c^2}
   &=&
   \dfrac{1}{2\norm{\Delta V_i}}\dfrac{\mathrm{d}^2\norm{\Delta V_i}^2}{\mathrm{d}\omega_c^2}
   -
   \dfrac{1}{4\norm{\Delta V_i}^3}
   \left(\dfrac{\mathrm{d}\norm{\Delta V_i}^2}{\mathrm{d}\omega_c}\right)^2
\end{EQ}
%%%
with values
%%%
%%%
\begin{EQ}[rcl]
   \dfrac{1}{2}\dfrac{\mathrm{d}}{\mathrm{d}\omega_c}\norm{\Delta V_0}^2 &=& 
   %\omega_c - \dfrac{K^2}{\omega_c^3} 
   %-(\bm{v}_0\cdot \hat{\bm{c}})
   %+\dfrac{K}{\omega_c^2} (\bm{v}_0\cdot\hat{\bm{r}}_0)
   %=
   \omega_c-(\bm{v}_0\cdot \hat{\bm{c}})
   +\dfrac{K}{\omega_c^2}\left(\bm{v}_0\cdot\hat{\bm{r}}_0-\dfrac{K}{\omega_c}\right)
   \\
   \dfrac{1}{2}\dfrac{\mathrm{d}}{\mathrm{d}\omega_c}\norm{\Delta V_1}^2 &=&
   %\omega_c - \dfrac{K^2}{\omega_c^3}
   %-(\bm{v}_1\cdot \hat{\bm{c}})
   %-\dfrac{K}{\omega_c^2} (\bm{v}_1\cdot\hat{\bm{r}}_1)
   %=
   \omega_c-(\bm{v}_1\cdot \hat{\bm{c}})
   -\dfrac{K}{\omega_c^2}\left(\bm{v}_1\cdot\hat{\bm{r}}_1+\dfrac{K}{\omega_c}\right)
   \\
   \dfrac{1}{2}\dfrac{\mathrm{d}^2}{\mathrm{d}\omega_c^2}\norm{\Delta V_0}^2 &=& 
   %1 + 3 \dfrac{K^2}{\omega_c^4} 
   %-2\dfrac{K}{\omega_c^3} (\bm{v}_0\cdot\hat{\bm{r}}_0)
   %=
   1+\dfrac{K}{\omega_c^3}\left(3\dfrac{K}{\omega_c}-2\,\bm{v}_0\cdot\hat{\bm{r}}_0\right)
   \\
   \dfrac{1}{2}\dfrac{\mathrm{d}^2}{\mathrm{d}\omega_c^2}\norm{\Delta V_1}^2 &=&
   %1+3\dfrac{K^2}{\omega_c^4}
   %+2\dfrac{K}{\omega_c^3} (\bm{v}_1\cdot\hat{\bm{r}}_1)
   %=
   1+\dfrac{K}{\omega_c^3}\left(3\dfrac{K}{\omega_c}+2\,\bm{v}_1\cdot\hat{\bm{r}}_1\right)
\end{EQ}
%%%
and finally
%%%
\begin{EQ}
  \dfrac{\mathrm{d}\Delta V}{\mathrm{d}\omega_c}=0,
  \qquad\Rightarrow\qquad
  \dfrac{1}{\norm{\Delta V_0}}\dfrac{\mathrm{d}\norm{\Delta V_0}^2}{\mathrm{d}\omega_c}
  =
  -\dfrac{1}{\norm{\Delta V_1}}\dfrac{\mathrm{d}\norm{\Delta V_1}^2}{\mathrm{d}\omega_c}
  \\
  \Rightarrow\qquad
  \norm{\Delta V_1}^2\left(\dfrac{\mathrm{d}\norm{\Delta V_0}^2}{\mathrm{d}\omega_c}\right)^2
  -
  \norm{\Delta V_0}^2\left(\dfrac{\mathrm{d}\norm{\Delta V_1}^2}{\mathrm{d}\omega_c}\right)^2
  =0
\end{EQ}
%%%
The condition zero derivative results in a polynomial in $\omega_c$ of degree $8$
with coefficients:
%%%
\newcommand{\crz}{t_0}
\newcommand{\cru}{t_1}
\newcommand{\cvz}{t_2}
\newcommand{\cvu}{t_3}
\newcommand{\vzrz}{t_4}
\newcommand{\vuru}{t_5}
\begin{EQ}[rclrclrcl]
  \crz  &=& \hat{\bm{c}}\cdot\hat{\bm{r}_0},\quad&
  \cru  &=& \hat{\bm{c}}\cdot\hat{\bm{r}_1},\quad&
  \cvz  &=& \hat{\bm{c}}\cdot\bm{v}_0,\\
  \cvu  &=& \hat{\bm{c}}\cdot\bm{v}_1,\quad&
  \vzrz &=& \bm{v}_0\cdot\hat{\bm{r}}_0,\quad&
  \vuru &=& \bm{v}_1\cdot\hat{\bm{r}}_1,
\end{EQ}
\begin{EQ}[rcl]
  a_0 &=&
  2(\crz+\cru) {K}^{5}+ \left( {v_0}^{
2}-{v_1}^{2}-{\vzrz}^{2}+{\vuru}^{2} \right) {K}^{4}
  \\
  a_1 &=&
  4\left(\crz\vuru-\cru\vzrz-\cvz+\cvu\right) {K}^{4}+
  2\left( {v_0}^{2}\vuru+{v_1}^{2}\vzrz-\vuru{\vzrz}^{2}-{\vuru}^{2}\vzrz \right) {K}^{3}
\\
a_2 &=&
2\left(\crz{\vuru}^{2}+\cru{\vzrz}^{2}+\cvz\vzrz-4(\cvz\vuru+\vzrz\cvu)+\vuru\cvu\right) {K}^{3}
\\
&+& \left( {v_0}^{2}{
\vuru}^{2}-{v_1}^{2}{\vzrz}^{2} \right) {K}^{2}
\\
a_3 &=&
4\left(\crz\cvu+\cru\cvz+\vzrz+\vuru\right) {K}^{3} \\
&+& 
2\left( \left( -{v_1}^{2}+2{\vzrz}\vuru-{\vuru}^{2}\right) \cvz+
        \left( {v_0}^{2}+{\vzrz}^{2}-2\vzrz\vuru\right) \cvu \right) {K}^{2}
\\
a_4 &=&
\left( 4(\crz\vuru\cvu-\cru\vzrz\cvz)+2(v_1^{2}-v_0^{2})-{\cvz}^{2}+{\cvu}^{2}-{\vzrz
}^{2}+{\vuru}^{2} \right) {K}^{2}
\\
&+& 2\left({v_0}^{2}\cvu\vuru+2v_1^{2}\vzrz\cvz \right) K-4\left(\crz+\cru\right) {K}^{3}
\\
a_5 &=&
4\left( -\crz\vuru+\cru\vzrz+\cvz-\cvu\right) {K}^{2} \\
&-& 2\left(\vuru\cvz^{2}+ 2\left(\vzrz+\vuru\right) \cvu\cvz+{v_0}^{2}\vuru+{v_1}^{2}\vzrz
    +\vzrz{{\cvu}}^{2} \right) K
\\
a_6 &=&
2\left( \cru\cvz^{2}+ \left(\vzrz+4\,{\vuru} \right) \cvz
+\crz\cvu^{2}+ \left( 4\vzrz+\vuru\right) \cvu \right) K+{v_0}^{2}{{\cvu}}^{2}-{v_1}^{2}{\cvz}^{2}
\\
a_7 &=&
-4\left(\crz\cvu+\cru\cvz+\vzrz\vuru\right) K+2\,{\cvz}^{2}\cvu+ 
2\left({v_1}^{2}-{\cvu}^{2}\right) \cvz-2\,{v_0}^{2}
{\cvu}
\\
a_8 &=&
2\left( \crz+\cru\right) K+{v_0}^{2}-{v_1}^
{2}-{\cvz}^{2}+{\cvu}^{2}
\end{EQ}
%%%
real roots that satisfy first and second order conditions are candidate
for the minimum.
Given $s=(r_0+r_1+c)/2$ parameter $p$ (derived from $\omega_s$) 
must be searched in the interval $[p^-,p_m]$ for the ``long`` trajectory,
and in the interval $[p_m,p^+]$ for the ``short`` trajectory,
%%%
\begin{EQ}
   p^{\pm}=\dfrac{2}{c^2}(s-r_0)(s-r_1)
   \left[ 2s-c\pm2\sqrt{s(s-c)}\right],
   \quad
   p_m = \dfrac{2}{c}(s-r_0)(s-r_1)
\end{EQ}
%%%
from~\eqref{eq:omegac} a limitation for the real roots of the 
polynomial is:
%%%
\begin{EQ}
  s \in ......
\end{EQ}


\section{Invariants}

\begin{itemize}
  \item Orbit energy $v^2/2-\mu/r=-\mu/(2a)=-\mu(1-e^2)/(2p)$
\end{itemize}


\section{Impulsive maneuver}

Variazione parametri orbitali sotto spinta impulsiva,
approssimo $\bm{y}(t)=\bm{y}(0)+\bm{y}'(0)t+\mathcal{O}(t^2)$
e integro sotto la spinta costante $\bm{T}/m=\bm{J}/\Delta t$ nell'intervallo
$[0,\Delta t]$.
%%%
\begin{EQ}
  \Delta t\bm{y}'(0) 
  \approx
  \int_0^{\Delta t}\bm{A}(\bm{y}(0)+\bm{y}'(0)t)\frac{\bm{J}}{\Delta t}\mathrm{d}t
  +\Delta t \bm{b}
\end{EQ}
%%%
chiamo $\bm{B}(\bm{y})=\bm{A}(\bm{y})\bm{J}$ e $\Delta\bm{y}=\Delta t\bm{y}'$
%%%
\begin{EQ}[rcl]
  \int_0^{\Delta t}\bm{A}(\bm{y}(0)+\bm{y}'(0)t)\frac{\bm{J}}{\Delta t}\mathrm{d}t
  &\approx&
  \dfrac{1}{\Delta t}\int_0^{\Delta t}\bm{B}(\bm{y}(0)+\dfrac{\partial\bm{B}(\bm{y}(0))}{\partial\bm{y}}\bm{y}'(0)t\mathrm{d}t
  \\
  &\approx&\bm{B}(\bm{y}(0))+\dfrac{\partial\bm{B}(\bm{y}(0))}{\partial\bm{y}}\bm{y}'(0)\dfrac{\Delta t}{2}
  \\
  &\approx&\bm{B}(\bm{y}(0))+\dfrac{1}{2}\dfrac{\partial\bm{B}(\bm{y}(0))}{\partial\bm{y}}\Delta\bm{y}
\end{EQ}
%%%
per cui abbiamo la relazione sui salti delle coordinate equinoziali
%%%
\begin{EQ}
  \left(\bm{I}-\dfrac{1}{2}\dfrac{\partial}{\partial\bm{y}}\bm{A}(\bm{y}(0))\bm{J}\right)\Delta\bm{y}
  =\bm{A}(\bm{y}(0))\bm{J}
\end{EQ}
%%%



La prima manovra sposta il razzo madre dalla terra ad un cluster di asteroidi.
Si può formulare un questo modo: (caso 2 soli impulsi)\\

\begin{itemize}
  \item $\bm{r}_E(t)$ posizione della terra.
  \item $\bm{v}_E(t)$ velocità della terra.
  \item $p_0$, $f_0$, $g_0$, $h_0$, $k_0$ parametri equinoziali della terra con 
        $L_0=L(t_0)$ posizione sulla traiettoria
  \item $p_1$, $f_1$, $g_1$, $h_1$, $k_1$ parametri equinoziali del cluster
\end{itemize}

Bisogna trovare i parametri equinoziali $p$, $f$, $g$, $h$, $k$
che danno la traiettoria del razzo nella tratto ``senza motore'' che 
soddisfano le equazioni:
%%%
\begin{EQ}[rcl]
   \bm{r}(p, f, g, h, k, L_0) &=& \bm{r}_E(t_0) \\
   \bm{v}(p, f, g, h, k, L_0) &=& \bm{v}_E(t_0)+(v_{\infty}+\Delta v_0)\bm{d}
\end{EQ}
%%%
cioè parto dalla terra e uso un po di motore e l'eccesso di velocità che ci regalano.
Messe così le equazioni sono incasinate, sotto una spinta impulsiva
dalla~\eqref{eq:equi} il cambiamento dei parametri equinoziali dovrebbe valere
%%%
\begin{EQ}[rcl]
  \bm{y}_+ &=& \bm{y}_-\bm{A}(\bm{y}_-)\frac{\bm{J}}{m}
\end{EQ}
%%%
bisogna verificare che $\bm{r}(\bm{y}_+) = \bm{r}(\bm{y}_-)$...[DA FINIRE]


\begin{EQ}
   L_1 = L_0+\Delta t\sqrt{\mu p}\left(\dfrac{q}{p}\right)^{2}
\end{EQ}


\section{Asteroid position}

\paragraph{position}
\begin{EQ}[rcl]
  x &=& \dfrac{p( (1+h^2-k^2)\cos(L)+2hk\sin(L))}
              {(1+f\cos(L)+g\sin(L))(1+h^2+k^2)}
  \\
  y &=& \dfrac{p( (1-h^2+k^2)\sin(L)+2hk\cos(L))}
              {(1+f\cos(L)+g\sin(L))(1+h^2+k^2)}
  \\
  z &=& \dfrac{2p(h\sin(L)-k\cos(L))}
              {(1+f\cos(L)+g\sin(L))(1+h^2+k^2)}
\end{EQ}
\paragraph{velocity}
\begin{EQ}[rcl]
  v_x &=&
  \sqrt{\dfrac{\mu}{p}}\,
  \dfrac{2hk(\cos(L)+f) - (1+h^2-k^2)(\sin(L)+g)}{1+h^2+k^2} 
  \\
  v_y &=&
  \sqrt{\dfrac{\mu}{p}}\,
  \dfrac{(1-h^2+k^2)(\cos(L)+f) - 2hk(\sin(L)+g)}{1+h^2+k^2}
  \\
  v_z &=&
  2\sqrt{\dfrac{\mu}{p}}\,
  \dfrac{h(\cos(L)+f) + k(\sin(L)+g)}{1+h^2+k^2}
\end{EQ}

\begin{EQ}
  a = \dfrac{p}{1-e^2},
  \qquad
  M(t) = M_0 + (t-t_0)\sqrt{\dfrac{\mu}{a^3}},
\end{EQ}
%%%
solve
%%%
\begin{EQ}
  E(t)-e\sin(E(t))=M(t),\qquad e\sinh(F(t))-F(t)=M(t)
\end{EQ}
%%%
\begin{EQ}
  \tan\frac{\nu(t)}{2} = \frac{\sqrt{1+e}\sin(E(t)/2)}{\sqrt{1-e}\cos(E(t)/2)},
  \qquad
  L(t) = \nu(t) + \omega + \Omega= \nu(t)+\arctan\dfrac{g}{f} 
\end{EQ}
%%%
\begin{EQ}
  \sqrt{1-e}\sin\frac{\nu(t)}{2}\cos\frac{E(t)}{2} = \sqrt{1+e}\sin\frac{E(t)}{2}\cos\frac{\nu(t)}{2},
\end{EQ}

\paragraph{From $L(t)$ to $M(t)$}

\begin{EQ}[rcl]
   \nu(t) &=& L(t)-\arctan\dfrac{g}{f}
   \\
   E(t) &=&2\arctan\frac{\sin(\nu(t)/2)\sqrt{1-e}}{\cos(\nu(t)/2)\sqrt{1+e}}
   \\
   M(t) &=&E(t)-e\sin(E(t)),
\end{EQ}

\paragraph{Condition for $L(t_1)-L(t_0)$}
\begin{EQ}[rcl]
   L(t_1)-L(t_0) &=& \nu(t_1)-\nu(t_0)
   \\
   \dfrac{\sqrt{1-e}}{\sqrt{1+e}}\sin\frac{\nu(t_0)}{2}\cos\frac{E(t_0)}{2} 
   &=& \sin\frac{E(t_0)}{2}\cos\frac{\nu(t_0)}{2},
   \\
   \dfrac{\sqrt{1-e}}{\sqrt{1+e}}\sin\frac{\nu(t_1)}{2}\cos\frac{E(t_1)}{2} 
   &=& \sin\frac{E(t_1)}{2}\cos\frac{\nu(t_1)}{2},
   \\
   E(t_1)-E(t_0)-e(\sin(E(t_1))-\sin(E(t_0)))&=&
   (t_1-t_0)\sqrt{\dfrac{\mu}{a^3}},
\end{EQ}


\end{document}             
